
%SDOF_eqs
\begin{align}
\Delta_D &= \frac{\sum{M_i \delta_{ls,i}^2}}{\sum{M_i \delta_{ls,i}}}\\
M_{eff} &= \frac{\sum{M_i \delta_{ls,i}}}{\Delta_D}\\
H_{eff} &=  \frac{\sum{M_i \delta_{ls,i} h_i}}{\sum{M_i \delta_{ls,i}}}
\end{align}

%theta_y
\begin{align}
\theta_y &=0.5\frac{f_{ye}}{E_s}\frac{L_b}{D_b}
\end{align}

%Delta_y
\begin{align}
\Delta_y &= \\theta_y*H_{eff}
\end{align}

% Moment equilibrium


\subsection{Calculate element design forces}
The element forces were determined based on the equilibrium method.
The set inflection point on the first storey columns was used to determine the
column base moments ($M_{col}_{0}$) \\ref{eq:Mbase}.
The remaining overturning moment was then taken by the axial component ($T$) of the outer columns \\ref{eq:outerCol}.
The axial load was then distributed into the beams ($V_{beam}$) in proportion to the
storey shear force \ref{eq:Beam shear} however for ease of design and construction
the storeys forces were calcualted in groups of XXXXX

 \\ref{eq:groupbeam} and assuming an inflection point at half
the beam length the beam moments were determined \ref{eq:beam moments}.
\begin{align}
\sum{M_{col}_{0}} &=0.6 V_{base} H_0 \label{eq:Mbase} \\
 M_{overturning} &= \sum{F_i*h_i} \\\ \n T &= \\frac{M_{overturning}-
 \sum{M_{col}_{0}}}{\sum{bay_{lengths}}} \label{eq:outerCol} \\
   V_{beam}_{(i)} &=T \\frac{V_{i}}{\sum{V}} \label{eq:Beam shear} \\
    V_{beam}_{(i \& i+1)} = T \\frac{\\frac{V_{i}+...+V_{i+ XXXX}}{' + str(BO.beam_group_size) + '}}{\sum{V}} \label{eq:groupbeam} \\
     Moment_{beam}_i = 0.5 V_{beam}_i*Length_{span}
     \end{align}
